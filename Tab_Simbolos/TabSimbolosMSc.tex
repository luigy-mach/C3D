\chapter*{Lista de símbolos}
\addcontentsline{toc}{chapter}{\numberline{}Lista de símbolos}
\begin{comment}
Esta sección se incluyen símbolos generales (con letras latinas y griegas), subíndices, superíndices y abreviaturas (incluir sólo las clases de símbolos que se utilicen). Cada una de estas listas debe estar ubicada en orden alfabético de acuerdo con la primera letra del símbolo.
\end{comment}


\begin{comment}

\section*{Símbolos con letras latinas}
 \label{simbolos}
 \renewcommand{\arraystretch}{1.3}
\begin{longtable}[l]{>{$}l<{$}l>{$}l<{$}>{$}l<{$}}
%\begin{tabular}
\textbf{Símbolo}&\textbf{Término}&\textbf{Unidad SI}&\textbf{Definición}\\[0.5ex]\hline
\endfirsthead%
\textbf{Símbolo}&\textbf{Término}&\textbf{Unidad SI}&\textbf{Definición}\\[0.5ex]\hline
\endhead%
      A              &Área                                   &\text{m}^{2}                         &\int\int dxdy\\%
      A_{\text{BET}} &Área interna del sólido                &\frac{\text{m}^{2}}{\text{g}}        &\text{ver DIN ISO 9277}\\%
      A_{\text{g}}   &Área transversal de la fase gaseosa    &\text{m}^{2}                         &\text{Ec...}\\%
      A_{\text{s}}   &Área transversal de la carga a granel  &\text{m}^{2}                         &\text{Ec...}\\%
      a              &Coeficiente                            &1                                    &\text{Ec...}\\%
      a              &Contenido de ceniza                    &1                                    &\frac{m_{\text{ceniza}}}{m_{\text{bm,0}}}\\%
      c              &Contenido de carbono                   &1                                    &\frac{m_{\text{C}}}{m}\\%
      c              &Longitud de la cuerda                  &\text{m}                             &\text{Figura...}\\
      c              &Concentración de la cantidad de materia&\frac{\text{mol}}{\text{m}^{3}}      &\frac{n}{V}\\%
      D              &Diámetro                               &\text{m}                             &\\%
      E_{\text{A}}   &Energía de activación                  &\frac{\text{kJ}}{\text{mol}}         &\text{Ec....}\\%
      F              &Fracción de materia volátil            &1                                    &\text{ver DIN 51720}\\%
      Fr             &N\'{u}mero de Froude                       &1                                    &\frac{\omega^{2}R}{g_{\text{0}}}\\%
      \overrightarrow{g}&Aceleración de la gravedad          &\frac{\text{m}}{\text{s}^{2}}        &\frac{d^{2}\overrightarrow{r}}{dt^{2}}\\%
      H              &Entalpía                               &\text{J}                             &U+PV\\%
      H_{\text{o}}   &Poder calorífico superior              &\frac{\text{MJ}}{\text{kg}}          &\text{ver DIN 51857}\\%
      h              &Contenido de hidrógeno                 &1                                    &\frac{m_{\text{H}}}{m}\\%
      K              &Coeficiente de equilibrio              &1                                    &\text{Ec...}\\%
      L              &Longitud                               &\text{m}                             &DF\\%
      L              &Longitud del reactor                   &\text{m}                             &\text{Figura...}\\%
      m              &Masa                                   &\text{kg}                            &DF\\%
      \dot{m}        &Flujo de masa                          &\frac{\text{kg}}{\text{s}}           &\frac{m}{t}\\%
      n              &Velocidad de rotación                  &\frac{\text{1}}{\text{s}}            &\frac{\omega}{2\pi}\\%
      n              &Cantidad de materia                    &\text{mol}                           &DF\\%
      P              &Presión                                &\text{Pa}                            &\frac{\vec{F}\cdot\vec{n}}{A}\\%
      Q              &Calor                                  &\text{kJ}                            &\text{1. $LT$}\\%
      T              &Temperatura                            &\text{K}                             &DF\\%
      t              &Tiempo                                 &\text{s}                             &DF\\%
      x_{\text{i}}   &Fracción de la cantidad de materia     &1                                    &\frac{n_{\text{i}}}{n}\\%
      V              &Volumen                                &\text{m}^{3}                         &\int{dr^{3}}\\%
      \vec{u}        &Velocidad                              &\frac{\text{m}}{\text{s}}            &(\frac{dr}{dt},r\frac{d\upsilon}{dt},\frac{dz}{dt})\\%
      w_{\text{i}}   &Fracción en masa del componente i      &1                                    &\frac{m_{\text{i}}}{m_{\text{0}}}\\%
      w_{\text{w,i}} &Contenido de humedad de la sustancia i &1                                    &\frac{m_{\text{\wasser}}}{m_{\text{i,0}}}\\%
      Z              &Factor de gases reales                 &1                                    &\frac{pv}{RT}\\%
\end{longtable}
\vspace{5ex}
\end{comment}

\begin{comment}


\section*{Símbolos con letras griegas}

\begin{longtable}[l]{>{$}l<{$}l>{$}l<{$}>{$}l<{$}}
\textbf{Símbolo}&\textbf{Término}&\textbf{Unidad SI}&\textbf{Definición}\\[0.5ex] \hline%
\endfirsthead%
\textbf{Símbolo}&\textbf{Término}&\textbf{Unidad SI}&\textbf{Definición}\\[0.5ex] \hline%
\endhead%
\renewcommand{\arraystretch}{1.3}
 \label{simbolosg}
     \alpha_{\text{BET}}  &Factor de superficie                  &\frac{\text{m}^{2}}{\text{g}}   &(w_{\text{F,waf}})(A_{\text{BET}})\\%
     \beta_{\text{i}}     &Grado de formación del componente i   &1                               &\frac{m_{\text{i}}}{m_{\text{bm,0}}}\\%
     \gamma               &Wandhaftreibwinkel (Stahlblech)       &1                               &\text{Sección...}\\
     \epsilon             &Porosidad de la partícula             &1                               &1-\frac{\rho_{\text{s}}}{\rho_{\text{w}}}\\%
     \eta                 &mittlere Bettneigungswinkel (St\"{u}rzen) &1                               &\text{Figura...}\\%
     \theta               &Ángulo de inclinación de la cama      &1                               &\text{Figura...}\\
     \theta_{\text{O}}    &Ángulo superior de avalancha          &1                               &\text{Figura...}\\
     \theta_{\text{U}}    &Ángulo inferior de avalancha          &1                               &\text{Figura...}\\
     \kappa               &Velocidad de calentamientoe           &\frac{\text{K}}{\text{s}}       &\frac{dT}{dt}\\%
     \nu                  &Coeficiente estequiométrico           &1                               &\text{ver DIN 13345}\\%
     \rho_{\text{b}}      &Densidad a granel                     &\frac{\text{kg}}{\text{m}^{3}}  &\frac{m_{\text{S}}}{V_{\text{S}}}\;(\text{Sección...})\\
     \rho_{\text{s}}      &Densidad aparente                     &\frac{\text{kg}}{\text{m}^{3}}  &\frac{m_{\text{F}}}{V_{\text{P}}}\;(\text{Sección...})\\
     \rho_{\text{w}}      &Densidad verdadera                    &\frac{\text{kg}}{\text{m}^{3}}  &\frac{m_{\text{F}}}{V_{\text{F}}}\;(\text{Sección...})\\
     \tau                 &Tiempo adimensional                   &1                               &\text{Ec....}\\%
     \Phi_{\text{V}}      &Flujo volumétrico                     &\frac{\text{m}^{3}}{\text{s}}   &\frac{\Delta V}{\Delta t}\\
     \omega               &Velocidad angular                     &\frac{1}{\text{s}}              &\frac{d\upsilon}{dt}\\

\end{longtable}
\end{comment}

\begin{comment}

\section*{Subíndices}
\begin{longtable}[l]{>{}l<{}l}
  \textbf{Subíndice} & \textbf{Término} \\[0.5ex] \hline%
  \endfirsthead%
  \textbf{Subíndice} & \textbf{Término} \\[0.5ex] \hline%
  \endhead%
\renewcommand{\arraystretch}{1.4}\label{simbolosg}

 bm&materia orgánica\\%
 DR&Dubinin-Radushkevich\\%
 E&Experimental\\%
 g&Fase gaseosa\\%
 k&Condensado\\%
 Ma&Macroporos\\%
 P&Partícula\\%
 p&Poro\\%
 p&Pirolizado\\%
 R&Reacción\\%
 t&Total\\%
 wf&Libre de agua\\%
 waf&Libre de agua y de ceniza\\%
 0&Estado de referencia\\%

\end{longtable}

\setlength{\extrarowheight}{0pt}
\end{comment}

\begin{comment}

\section*{Superíndices}
\begin{longtable}[l]{>{}l<{}l}
  \textbf{Superíndice} & \textbf{Término} \\[0.5ex] \hline%
  \endfirsthead%
  \textbf{Superíndice} & \textbf{Término} \\[0.5ex] \hline%
  \endhead%
\renewcommand{\arraystretch}{1.4}\label{simbolosg}

 n &Coeficiente x\\%



\end{longtable}


\setlength{\extrarowheight}{0pt}
\end{comment}

\section*{Abreviaturas}
\begin{longtable}[l]{>{}l<{}l}
  \textbf{Abreviatura} & \textbf{Término} \\[0.5ex] \hline%
  \endfirsthead%
  \textbf{Abreviatura} & \textbf{Término} \\[0.5ex] \hline%
  \endhead%
\renewcommand{\arraystretch}{1.4}
\label{abreviaturas}

$ML$	 &	Machine Learning	\\%
$DF$    &Dimensión fundamental				\\%
$HT$  &Hough Transform	\\%
$HCS$ &Histograms of Categorized Shapes	\\%
$PCA$ &Principal Component Analysis	\\%
$GFD$ &Generic Fourier Descriptor	\\%
$IA$  &Inteligencia Artificial	\\%
$CV$  &Computer vision	\\%
\end{longtable}

\listoffigures
%\listoftables



\setlength{\extrarowheight}{0pt}