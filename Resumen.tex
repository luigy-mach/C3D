\thispagestyle{empty} \vspace*{10ex} \textbf{\centerline{\LARGE
Abstract}}\normalsize\\\\\\%
El reconocimiento de movimiento y acciones para video-vigilancia es un campo activo de visión computacional. Hoy en dìa, existen varias técnicas que intentan abordar este tema, algunos mediante, mapeo 3D con un alto costo computacional. Este artículo describe algoritmos de software que pueden detectar a las personas en la escena y analizar diferentes acciones y patrones de movimiento en tiempo real.\\
 
La motivación para este trabajo es crear un sistema capaz de poder acciones violentas en videos de vigilancia, sin la necesidad de asistencia humana para la detección de dichos actos.\\
 
Utilizamos un método para la segmentación del primer plano y del fondo, crearemos un vector caracteristico para discriminar y seguir a varias personas de la escena. Finalmente, se describe un simple algoritmo basado en MVFI para discriminar acciones y movimientos corporales a través de la magnitud de la velocidad y la dirección en la cual se detecta el movimiento de píxeles.


\newpage
%%%%%%%%%%%%%%%%%%%%%%%%%%%%%%%%

\begin{comment}


ESTRUCTURA  TESIS \\
------------------\\
cap1 - introducción

\begin{itemize}
\item concepto y motivación
\item descripción del problema
\item objetivos
  \begin{itemize}
  \item generales
  \item especificos
  \end{itemize}
\item estructura del documento
\end{itemize}

cap2 - trabajos relacionados
\begin{itemize}
\item consideraciones iniciales
\item descripción de trabajos
\item comparación de trabajos
\item consideraciones finales
\end{itemize}

cap3 marco teorico
-------
\begin{itemize}
\item no sé q irá aqui...
\end{itemize}
\end{comment}

